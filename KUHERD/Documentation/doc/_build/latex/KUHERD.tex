%% Generated by Sphinx.
\def\sphinxdocclass{report}
\documentclass[letterpaper,10pt,english]{sphinxmanual}
\ifdefined\pdfpxdimen
   \let\sphinxpxdimen\pdfpxdimen\else\newdimen\sphinxpxdimen
\fi \sphinxpxdimen=.75bp\relax

\usepackage[utf8]{inputenc}
\ifdefined\DeclareUnicodeCharacter
 \ifdefined\DeclareUnicodeCharacterAsOptional\else
  \DeclareUnicodeCharacter{00A0}{\nobreakspace}
\fi\fi
\usepackage{cmap}
\usepackage[T1]{fontenc}
\usepackage{amsmath,amssymb,amstext}
\usepackage{babel}
\usepackage{times}
\usepackage[Sonny]{fncychap}
\usepackage{longtable}
\usepackage{sphinx}

\usepackage{geometry}
\usepackage{multirow}
\usepackage{eqparbox}

% Include hyperref last.
\usepackage{hyperref}
% Fix anchor placement for figures with captions.
\usepackage{hypcap}% it must be loaded after hyperref.
% Set up styles of URL: it should be placed after hyperref.
\urlstyle{same}
\addto\captionsenglish{\renewcommand{\contentsname}{Contents:}}

\addto\captionsenglish{\renewcommand{\figurename}{Fig.}}
\addto\captionsenglish{\renewcommand{\tablename}{Table}}
\addto\captionsenglish{\renewcommand{\literalblockname}{Listing}}

\addto\extrasenglish{\def\pageautorefname{page}}

\setcounter{tocdepth}{3}
\setcounter{secnumdepth}{3}


\title{KUHERD Documentation}
\date{May 31, 2017}
\release{1.0}
\author{Joseph St.Amand}
\newcommand{\sphinxlogo}{}
\renewcommand{\releasename}{Release}
\makeindex

\begin{document}

\maketitle
\sphinxtableofcontents
\phantomsection\label{\detokenize{index::doc}}



\chapter{KUHERD package}
\label{\detokenize{KUHERD:kuherd-package}}\label{\detokenize{KUHERD:welcome-to-kuherd-s-documentation}}\label{\detokenize{KUHERD::doc}}

\section{Subpackages}
\label{\detokenize{KUHERD:subpackages}}

\subsection{KUHERD.Experiments package}
\label{\detokenize{KUHERD.Experiments:kuherd-experiments-package}}\label{\detokenize{KUHERD.Experiments::doc}}

\subsubsection{Submodules}
\label{\detokenize{KUHERD.Experiments:submodules}}

\subsubsection{KUHERD.Experiments.MultiDT module}
\label{\detokenize{KUHERD.Experiments:module-KUHERD.Experiments.MultiDT}}\label{\detokenize{KUHERD.Experiments:kuherd-experiments-multidt-module}}\index{KUHERD.Experiments.MultiDT (module)}\index{MultiDT() (in module KUHERD.Experiments.MultiDT)}

\begin{fulllineitems}
\phantomsection\label{\detokenize{KUHERD.Experiments:KUHERD.Experiments.MultiDT.MultiDT}}\pysiglinewithargsret{\sphinxcode{KUHERD.Experiments.MultiDT.}\sphinxbfcode{MultiDT}}{}{}
Program for running an experiment using Decision Tree classifier.

\end{fulllineitems}



\subsubsection{KUHERD.Experiments.MultiLR module}
\label{\detokenize{KUHERD.Experiments:module-KUHERD.Experiments.MultiLR}}\label{\detokenize{KUHERD.Experiments:kuherd-experiments-multilr-module}}\index{KUHERD.Experiments.MultiLR (module)}\index{MultiLR() (in module KUHERD.Experiments.MultiLR)}

\begin{fulllineitems}
\phantomsection\label{\detokenize{KUHERD.Experiments:KUHERD.Experiments.MultiLR.MultiLR}}\pysiglinewithargsret{\sphinxcode{KUHERD.Experiments.MultiLR.}\sphinxbfcode{MultiLR}}{}{}
Program for running an experiment using Loigistic Regression classifier.

\end{fulllineitems}



\subsubsection{KUHERD.Experiments.MultiNB module}
\label{\detokenize{KUHERD.Experiments:kuherd-experiments-multinb-module}}\label{\detokenize{KUHERD.Experiments:module-KUHERD.Experiments.MultiNB}}\index{KUHERD.Experiments.MultiNB (module)}\index{MultiNB() (in module KUHERD.Experiments.MultiNB)}

\begin{fulllineitems}
\phantomsection\label{\detokenize{KUHERD.Experiments:KUHERD.Experiments.MultiNB.MultiNB}}\pysiglinewithargsret{\sphinxcode{KUHERD.Experiments.MultiNB.}\sphinxbfcode{MultiNB}}{}{}
Program for running an experiment using Naive Bayes classifier.

\end{fulllineitems}



\subsubsection{KUHERD.Experiments.MultiRF module}
\label{\detokenize{KUHERD.Experiments:kuherd-experiments-multirf-module}}\label{\detokenize{KUHERD.Experiments:module-KUHERD.Experiments.MultiRF}}\index{KUHERD.Experiments.MultiRF (module)}\index{MultiRF() (in module KUHERD.Experiments.MultiRF)}

\begin{fulllineitems}
\phantomsection\label{\detokenize{KUHERD.Experiments:KUHERD.Experiments.MultiRF.MultiRF}}\pysiglinewithargsret{\sphinxcode{KUHERD.Experiments.MultiRF.}\sphinxbfcode{MultiRF}}{}{}
Program for running an experiment using Random Forest classifier.

\end{fulllineitems}



\subsubsection{KUHERD.Experiments.MultiSVM module}
\label{\detokenize{KUHERD.Experiments:kuherd-experiments-multisvm-module}}\label{\detokenize{KUHERD.Experiments:module-KUHERD.Experiments.MultiSVM}}\index{KUHERD.Experiments.MultiSVM (module)}\index{MultiSVM() (in module KUHERD.Experiments.MultiSVM)}

\begin{fulllineitems}
\phantomsection\label{\detokenize{KUHERD.Experiments:KUHERD.Experiments.MultiSVM.MultiSVM}}\pysiglinewithargsret{\sphinxcode{KUHERD.Experiments.MultiSVM.}\sphinxbfcode{MultiSVM}}{}{}
Program for running an experiment using Support Vector Machine classifier.

\end{fulllineitems}



\subsubsection{Module contents}
\label{\detokenize{KUHERD.Experiments:module-KUHERD.Experiments}}\label{\detokenize{KUHERD.Experiments:module-contents}}\index{KUHERD.Experiments (module)}

\subsection{KUHERD.Tools package}
\label{\detokenize{KUHERD.Tools:kuherd-tools-package}}\label{\detokenize{KUHERD.Tools::doc}}

\subsubsection{Submodules}
\label{\detokenize{KUHERD.Tools:submodules}}

\subsubsection{KUHERD.Tools.makePredictions module}
\label{\detokenize{KUHERD.Tools:kuherd-tools-makepredictions-module}}\label{\detokenize{KUHERD.Tools:module-KUHERD.Tools.makePredictions}}\index{KUHERD.Tools.makePredictions (module)}\index{main() (in module KUHERD.Tools.makePredictions)}

\begin{fulllineitems}
\phantomsection\label{\detokenize{KUHERD.Tools:KUHERD.Tools.makePredictions.main}}\pysiglinewithargsret{\sphinxcode{KUHERD.Tools.makePredictions.}\sphinxbfcode{main}}{}{}
\end{fulllineitems}



\subsubsection{KUHERD.Tools.trainModel module}
\label{\detokenize{KUHERD.Tools:kuherd-tools-trainmodel-module}}\label{\detokenize{KUHERD.Tools:module-KUHERD.Tools.trainModel}}\index{KUHERD.Tools.trainModel (module)}\index{main() (in module KUHERD.Tools.trainModel)}

\begin{fulllineitems}
\phantomsection\label{\detokenize{KUHERD.Tools:KUHERD.Tools.trainModel.main}}\pysiglinewithargsret{\sphinxcode{KUHERD.Tools.trainModel.}\sphinxbfcode{main}}{}{}
\end{fulllineitems}



\subsubsection{Module contents}
\label{\detokenize{KUHERD.Tools:module-contents}}\label{\detokenize{KUHERD.Tools:module-KUHERD.Tools}}\index{KUHERD.Tools (module)}

\section{Submodules}
\label{\detokenize{KUHERD:submodules}}

\section{KUHERD.FeatureSelector module}
\label{\detokenize{KUHERD:kuherd-featureselector-module}}\label{\detokenize{KUHERD:module-KUHERD.FeatureSelector}}\index{KUHERD.FeatureSelector (module)}\index{FeatureSelector (class in KUHERD.FeatureSelector)}

\begin{fulllineitems}
\phantomsection\label{\detokenize{KUHERD:KUHERD.FeatureSelector.FeatureSelector}}\pysiglinewithargsret{\sphinxstrong{class }\sphinxcode{KUHERD.FeatureSelector.}\sphinxbfcode{FeatureSelector}}{\emph{scoring\_function}, \emph{kbest}}{}
Bases: \sphinxcode{object}
\index{fit() (KUHERD.FeatureSelector.FeatureSelector method)}

\begin{fulllineitems}
\phantomsection\label{\detokenize{KUHERD:KUHERD.FeatureSelector.FeatureSelector.fit}}\pysiglinewithargsret{\sphinxbfcode{fit}}{\emph{X}, \emph{Y}, \emph{label\_set}}{}
Fits the data by training the feature selection model compomnent.
\begin{description}
\item[{Args:}] \leavevmode
X (numpy matrix): The data matrix.
y (integer list): The labels for the data.
label\_set (str): Either `purpose' or `field'.

\item[{Return:}] \leavevmode
None

\end{description}

\end{fulllineitems}

\index{transform() (KUHERD.FeatureSelector.FeatureSelector method)}

\begin{fulllineitems}
\phantomsection\label{\detokenize{KUHERD:KUHERD.FeatureSelector.FeatureSelector.transform}}\pysiglinewithargsret{\sphinxbfcode{transform}}{\emph{X}}{}
Transforms the data, retaining only features learned in the \sphinxquotedblleft{}fit\sphinxquotedblright{} process.
\begin{description}
\item[{Args:}] \leavevmode
X (numpy matrix): The data matrix.
y (integer list): The labels for the data.

\item[{Return:}] \leavevmode
(numpy matrix): Transformed data matrix.

\end{description}

\end{fulllineitems}


\end{fulllineitems}



\section{KUHERD.HerdVectorizer module}
\label{\detokenize{KUHERD:module-KUHERD.HerdVectorizer}}\label{\detokenize{KUHERD:kuherd-herdvectorizer-module}}\index{KUHERD.HerdVectorizer (module)}\index{HerdVectorizer (class in KUHERD.HerdVectorizer)}

\begin{fulllineitems}
\phantomsection\label{\detokenize{KUHERD:KUHERD.HerdVectorizer.HerdVectorizer}}\pysiglinewithargsret{\sphinxstrong{class }\sphinxcode{KUHERD.HerdVectorizer.}\sphinxbfcode{HerdVectorizer}}{\emph{config}}{}
Bases: \sphinxcode{object}

Main class responsible for vectorization of the text data. This class is extremely configurable, with many
options for each preprocessing behavior, bigrams, stemmers, stopwords, and feature selection. Use of this
class is done in the following manner:
- Set configuration options for preproc\_config, bigram\_config, stemmer, stopwords, and feature selection.
- Train the vectorizer on a set of documents and their corresponding labels.
- After training is complete, the documents may be given to the transform function to convert to TFIDF form.
\index{create\_bigram\_index\_map() (KUHERD.HerdVectorizer.HerdVectorizer method)}

\begin{fulllineitems}
\phantomsection\label{\detokenize{KUHERD:KUHERD.HerdVectorizer.HerdVectorizer.create_bigram_index_map}}\pysiglinewithargsret{\sphinxbfcode{create\_bigram\_index\_map}}{\emph{tokenized\_docs}}{}
Creates a mapping from each bigram to a column index.
\begin{description}
\item[{Args:}] \leavevmode
tokenized\_docs (list): A list of documents, each document is represented as a single long string

\item[{Return:}] \leavevmode
(dictionary): Dictionary where keys are tokens, values are the index into a feature matrix.

\end{description}

\end{fulllineitems}

\index{create\_token\_index\_map() (KUHERD.HerdVectorizer.HerdVectorizer method)}

\begin{fulllineitems}
\phantomsection\label{\detokenize{KUHERD:KUHERD.HerdVectorizer.HerdVectorizer.create_token_index_map}}\pysiglinewithargsret{\sphinxbfcode{create\_token\_index\_map}}{\emph{tokenized\_docs}}{}
Given the tokenized documents, finds all unique tokens and forms an index map
\begin{description}
\item[{Args:}] \leavevmode
tokenized\_docs (list): A list of documents, each document is represented as a single long string

\item[{Return:}] \leavevmode
(dictionary): Dictionary where keys are tokens, values are the index into a feature matrix.

\end{description}

\end{fulllineitems}

\index{filter\_docs() (KUHERD.HerdVectorizer.HerdVectorizer static method)}

\begin{fulllineitems}
\phantomsection\label{\detokenize{KUHERD:KUHERD.HerdVectorizer.HerdVectorizer.filter_docs}}\pysiglinewithargsret{\sphinxstrong{static }\sphinxbfcode{filter\_docs}}{\emph{tokenized\_docs}, \emph{tok\_index\_map}}{}
Filters tokenized documents, removing all tokens which are not recognized by the specified token index map.
\begin{description}
\item[{Args:}] \leavevmode
tokenized\_docs (list): A list of documents, each document is represented as a single long string
tok\_index\_map (dictionary): A mapping from tokens to their index in the feature matrix.

\item[{Return:}] \leavevmode
(list): A list of documents, where each document is a list of (filtered) tokens.

\end{description}

\end{fulllineitems}

\index{form\_bigram\_count\_matrix() (KUHERD.HerdVectorizer.HerdVectorizer method)}

\begin{fulllineitems}
\phantomsection\label{\detokenize{KUHERD:KUHERD.HerdVectorizer.HerdVectorizer.form_bigram_count_matrix}}\pysiglinewithargsret{\sphinxbfcode{form\_bigram\_count\_matrix}}{\emph{tokenized\_docs}}{}
Calculates a bigram count matrix from a list of tokenized documents.
\begin{description}
\item[{Args:}] \leavevmode
tokenized\_docs (list): A list of documents, each document is represented as a single long string

\item[{Return:}] \leavevmode
(sparse numpy matrix): A sparse count matrix in COO format.

\end{description}

\end{fulllineitems}

\index{form\_count\_matrix() (KUHERD.HerdVectorizer.HerdVectorizer method)}

\begin{fulllineitems}
\phantomsection\label{\detokenize{KUHERD:KUHERD.HerdVectorizer.HerdVectorizer.form_count_matrix}}\pysiglinewithargsret{\sphinxbfcode{form\_count\_matrix}}{\emph{tokenized\_docs}}{}
Forms the count matrix from the tokenized documents
\begin{description}
\item[{Args:}] \leavevmode
tokenized\_docs (list): A list of lists representing the tokenized documents. Each document is a list of tokens.

\item[{Return:}] \leavevmode
(sparse numpy matrix): A sparse count matrix in COO format.

\end{description}

\end{fulllineitems}

\index{getConfig() (KUHERD.HerdVectorizer.HerdVectorizer method)}

\begin{fulllineitems}
\phantomsection\label{\detokenize{KUHERD:KUHERD.HerdVectorizer.HerdVectorizer.getConfig}}\pysiglinewithargsret{\sphinxbfcode{getConfig}}{}{}
Retrieve the complete configuration needed to build a vectorizer.

Returns the complete configuration needed to build a vectorizer. Note that the config returned may only be used
to train a new vectorizer. The config does NOT give model persistance.

\end{fulllineitems}

\index{get\_bigram\_config() (KUHERD.HerdVectorizer.HerdVectorizer method)}

\begin{fulllineitems}
\phantomsection\label{\detokenize{KUHERD:KUHERD.HerdVectorizer.HerdVectorizer.get_bigram_config}}\pysiglinewithargsret{\sphinxbfcode{get\_bigram\_config}}{}{}
Retrieve the bigram configuration.

\end{fulllineitems}

\index{get\_preproc\_config() (KUHERD.HerdVectorizer.HerdVectorizer method)}

\begin{fulllineitems}
\phantomsection\label{\detokenize{KUHERD:KUHERD.HerdVectorizer.HerdVectorizer.get_preproc_config}}\pysiglinewithargsret{\sphinxbfcode{get\_preproc\_config}}{}{}
Retrieve the preprocessor configuration.

\end{fulllineitems}

\index{lancaster\_stemmer() (KUHERD.HerdVectorizer.HerdVectorizer static method)}

\begin{fulllineitems}
\phantomsection\label{\detokenize{KUHERD:KUHERD.HerdVectorizer.HerdVectorizer.lancaster_stemmer}}\pysiglinewithargsret{\sphinxstrong{static }\sphinxbfcode{lancaster\_stemmer}}{\emph{docs}}{}
Lancaster stemming algorithm

\end{fulllineitems}

\index{porter\_stemmer() (KUHERD.HerdVectorizer.HerdVectorizer static method)}

\begin{fulllineitems}
\phantomsection\label{\detokenize{KUHERD:KUHERD.HerdVectorizer.HerdVectorizer.porter_stemmer}}\pysiglinewithargsret{\sphinxstrong{static }\sphinxbfcode{porter\_stemmer}}{\emph{docs}}{}
Porter stemming algorithm

\end{fulllineitems}

\index{set\_bigram\_config() (KUHERD.HerdVectorizer.HerdVectorizer method)}

\begin{fulllineitems}
\phantomsection\label{\detokenize{KUHERD:KUHERD.HerdVectorizer.HerdVectorizer.set_bigram_config}}\pysiglinewithargsret{\sphinxbfcode{set\_bigram\_config}}{\emph{name}, \emph{value}}{}
Set the bigram configuration.

\end{fulllineitems}

\index{set\_bigrams() (KUHERD.HerdVectorizer.HerdVectorizer method)}

\begin{fulllineitems}
\phantomsection\label{\detokenize{KUHERD:KUHERD.HerdVectorizer.HerdVectorizer.set_bigrams}}\pysiglinewithargsret{\sphinxbfcode{set\_bigrams}}{\emph{bigrams}, \emph{bigram\_window\_size}, \emph{bigram\_filter\_size}, \emph{bigram\_nbest}}{}
Set the bigram configuration.

\end{fulllineitems}

\index{set\_feature\_selector() (KUHERD.HerdVectorizer.HerdVectorizer method)}

\begin{fulllineitems}
\phantomsection\label{\detokenize{KUHERD:KUHERD.HerdVectorizer.HerdVectorizer.set_feature_selector}}\pysiglinewithargsret{\sphinxbfcode{set\_feature\_selector}}{\emph{scoring\_func}, \emph{kbest}, \emph{multi\_type}}{}
Set the feature selection configuration values.

\end{fulllineitems}

\index{set\_preproc\_config() (KUHERD.HerdVectorizer.HerdVectorizer method)}

\begin{fulllineitems}
\phantomsection\label{\detokenize{KUHERD:KUHERD.HerdVectorizer.HerdVectorizer.set_preproc_config}}\pysiglinewithargsret{\sphinxbfcode{set\_preproc\_config}}{\emph{name}, \emph{value}}{}
Set the preprocessor configuration value.

\end{fulllineitems}

\index{set\_stemmer() (KUHERD.HerdVectorizer.HerdVectorizer method)}

\begin{fulllineitems}
\phantomsection\label{\detokenize{KUHERD:KUHERD.HerdVectorizer.HerdVectorizer.set_stemmer}}\pysiglinewithargsret{\sphinxbfcode{set\_stemmer}}{\emph{the\_stemmer}}{}
Set the stemmer configuration values.

\end{fulllineitems}

\index{snowball\_stemmer() (KUHERD.HerdVectorizer.HerdVectorizer static method)}

\begin{fulllineitems}
\phantomsection\label{\detokenize{KUHERD:KUHERD.HerdVectorizer.HerdVectorizer.snowball_stemmer}}\pysiglinewithargsret{\sphinxstrong{static }\sphinxbfcode{snowball\_stemmer}}{\emph{docs}}{}
Snowball stemming algorithm

\end{fulllineitems}

\index{tokenize\_docs() (KUHERD.HerdVectorizer.HerdVectorizer method)}

\begin{fulllineitems}
\phantomsection\label{\detokenize{KUHERD:KUHERD.HerdVectorizer.HerdVectorizer.tokenize_docs}}\pysiglinewithargsret{\sphinxbfcode{tokenize\_docs}}{\emph{docs}}{}
Breaks each document down into a list of words(tokens).

Converts a list of documents(each document is given as a single string) and converts them to their tokenized
form in the following manner(some steps may be skipped if configured as such in the configuration settings)
\begin{itemize}
\item {} 
break document into tokens

\item {} 
remove punctuation

\item {} 
stem tokens

\end{itemize}
\begin{description}
\item[{Args:}] \leavevmode
docs (list): A list of documents, each document is represented as a single long string

\item[{Return:}] \leavevmode
(list): The tokenized documents as a list of lists, each item of the outer list is a document, which is represented as a list of words.

\end{description}

\end{fulllineitems}

\index{train() (KUHERD.HerdVectorizer.HerdVectorizer method)}

\begin{fulllineitems}
\phantomsection\label{\detokenize{KUHERD:KUHERD.HerdVectorizer.HerdVectorizer.train}}\pysiglinewithargsret{\sphinxbfcode{train}}{\emph{docs}, \emph{y}, \emph{label\_set}}{}
Takes a list of documents, and the corresponding labels and trains the preprocessor(including feature selection).
\begin{description}
\item[{Args:}] \leavevmode
docs (list): A list of documents, where each document is represented as a string.
y (list): A list of integers representing the label for each document.
label\_set (str): Specifies the label set so that the input `y' may be interpreted. Valiud entries are either `purpose' or `field'.

\end{description}

@param docs The list of documents
@param y A vector of labels which correspond to each document

\end{fulllineitems}

\index{transform\_data() (KUHERD.HerdVectorizer.HerdVectorizer method)}

\begin{fulllineitems}
\phantomsection\label{\detokenize{KUHERD:KUHERD.HerdVectorizer.HerdVectorizer.transform_data}}\pysiglinewithargsret{\sphinxbfcode{transform\_data}}{\emph{docs}}{}
Tranforms documents into a sparse matrix.
\begin{description}
\item[{Must be called after the preprocessor has been trained on some data. Process is as follows:}] \leavevmode
-tokenize documents
-search for bigrams
-transform to TFIDF representation
-select features

\item[{Args:}] \leavevmode
docs (list): A list of documents, each document is represented as a string.

\item[{Return:}] \leavevmode
(sparse numpy matrix): A sparse CSR formatted matrix, each row corresponds to a document, ordering of documents is preserved.

\end{description}

\end{fulllineitems}


\end{fulllineitems}

\index{main() (in module KUHERD.HerdVectorizer)}

\begin{fulllineitems}
\phantomsection\label{\detokenize{KUHERD:KUHERD.HerdVectorizer.main}}\pysiglinewithargsret{\sphinxcode{KUHERD.HerdVectorizer.}\sphinxbfcode{main}}{}{}
\end{fulllineitems}



\section{KUHERD.LabelTransformer module}
\label{\detokenize{KUHERD:module-KUHERD.LabelTransformer}}\label{\detokenize{KUHERD:kuherd-labeltransformer-module}}\index{KUHERD.LabelTransformer (module)}\index{LabelTransformer (class in KUHERD.LabelTransformer)}

\begin{fulllineitems}
\phantomsection\label{\detokenize{KUHERD:KUHERD.LabelTransformer.LabelTransformer}}\pysiglinewithargsret{\sphinxstrong{class }\sphinxcode{KUHERD.LabelTransformer.}\sphinxbfcode{LabelTransformer}}{\emph{label\_set\_name}, \emph{labels}}{}
Bases: \sphinxcode{object}
\index{default\_labels() (KUHERD.LabelTransformer.LabelTransformer method)}

\begin{fulllineitems}
\phantomsection\label{\detokenize{KUHERD:KUHERD.LabelTransformer.LabelTransformer.default_labels}}\pysiglinewithargsret{\sphinxbfcode{default\_labels}}{\emph{target\_set}}{}
\end{fulllineitems}

\index{label2mat() (KUHERD.LabelTransformer.LabelTransformer method)}

\begin{fulllineitems}
\phantomsection\label{\detokenize{KUHERD:KUHERD.LabelTransformer.LabelTransformer.label2mat}}\pysiglinewithargsret{\sphinxbfcode{label2mat}}{\emph{x}}{}
Converts a vector of strings to a matrix of of zero-one valued columns.
\begin{description}
\item[{Args:}] \leavevmode
x (list): A vector containing string values representing the labels.

\item[{Return:}] \leavevmode
(numpy matrix): A matrix of zero-one valued columns.

\end{description}

\end{fulllineitems}

\index{mat2vec() (KUHERD.LabelTransformer.LabelTransformer method)}

\begin{fulllineitems}
\phantomsection\label{\detokenize{KUHERD:KUHERD.LabelTransformer.LabelTransformer.mat2vec}}\pysiglinewithargsret{\sphinxbfcode{mat2vec}}{\emph{M}}{}
Converts a zero-one valued label matrix to an integer valued label vector.
\begin{description}
\item[{Args:}] \leavevmode
M (numpy mat): A zero-one valued label matrix.

\end{description}

\end{fulllineitems}

\index{vec2mat() (KUHERD.LabelTransformer.LabelTransformer method)}

\begin{fulllineitems}
\phantomsection\label{\detokenize{KUHERD:KUHERD.LabelTransformer.LabelTransformer.vec2mat}}\pysiglinewithargsret{\sphinxbfcode{vec2mat}}{\emph{x}}{}
Converts a vector of integers to a matrix of of zero-one valued columns.
\begin{description}
\item[{Args:}] \leavevmode
x (list): A vector containing integer values mapping to members of the label\_type.

\item[{Return:}] \leavevmode
(numpy matrix): A matrix of zero-one valued columns.

\end{description}

\end{fulllineitems}

\index{vec2string() (KUHERD.LabelTransformer.LabelTransformer method)}

\begin{fulllineitems}
\phantomsection\label{\detokenize{KUHERD:KUHERD.LabelTransformer.LabelTransformer.vec2string}}\pysiglinewithargsret{\sphinxbfcode{vec2string}}{\emph{label\_vec}}{}
Converts vector containing integers to a string representation using the label set dictionaries.
Args:
\begin{quote}

label\_vec (list): A vector containing integer values mapping to members of the label\_type.
\end{quote}
\begin{description}
\item[{Return:}] \leavevmode
(list): A list of strings that are members of the label\_type.

\end{description}

\end{fulllineitems}


\end{fulllineitems}



\section{KUHERD.Models module}
\label{\detokenize{KUHERD:kuherd-models-module}}\label{\detokenize{KUHERD:module-KUHERD.Models}}\index{KUHERD.Models (module)}\index{ClassificationModel (class in KUHERD.Models)}

\begin{fulllineitems}
\phantomsection\label{\detokenize{KUHERD:KUHERD.Models.ClassificationModel}}\pysiglinewithargsret{\sphinxstrong{class }\sphinxcode{KUHERD.Models.}\sphinxbfcode{ClassificationModel}}{\emph{config}}{}
Bases: \sphinxcode{object}
\index{fit() (KUHERD.Models.ClassificationModel method)}

\begin{fulllineitems}
\phantomsection\label{\detokenize{KUHERD:KUHERD.Models.ClassificationModel.fit}}\pysiglinewithargsret{\sphinxbfcode{fit}}{\emph{X}, \emph{Y}}{}
Trains the model.

Fitting or \sphinxquotedblleft{}training\sphinxquotedblright{} must be done before the model is able to make predictions.
\begin{description}
\item[{Args:}] \leavevmode
X (numpy matrix): Training samples.
Y (numpy matrix): Training labels.

\item[{Returns:}] \leavevmode
None: No return value.

\end{description}

\end{fulllineitems}

\index{get\_config() (KUHERD.Models.ClassificationModel method)}

\begin{fulllineitems}
\phantomsection\label{\detokenize{KUHERD:KUHERD.Models.ClassificationModel.get_config}}\pysiglinewithargsret{\sphinxbfcode{get\_config}}{}{}
Returns the configuration used to build this model.
\begin{description}
\item[{Returns:}] \leavevmode
dict: dictionary containing target label set, internal model configuration, and model name.

\end{description}

\end{fulllineitems}

\index{predict() (KUHERD.Models.ClassificationModel method)}

\begin{fulllineitems}
\phantomsection\label{\detokenize{KUHERD:KUHERD.Models.ClassificationModel.predict}}\pysiglinewithargsret{\sphinxbfcode{predict}}{\emph{X}}{}
Make predictions.
\begin{description}
\item[{Args:}] \leavevmode
X (numpy matrix): Training samples.

\item[{Returns:}] \leavevmode
numpy matrix: predicted label values.

\end{description}

\end{fulllineitems}


\end{fulllineitems}

\index{PurposeFieldModel (class in KUHERD.Models)}

\begin{fulllineitems}
\phantomsection\label{\detokenize{KUHERD:KUHERD.Models.PurposeFieldModel}}\pysiglinewithargsret{\sphinxstrong{class }\sphinxcode{KUHERD.Models.}\sphinxbfcode{PurposeFieldModel}}{\emph{config}}{}
Bases: \sphinxcode{object}
\index{fit() (KUHERD.Models.PurposeFieldModel method)}

\begin{fulllineitems}
\phantomsection\label{\detokenize{KUHERD:KUHERD.Models.PurposeFieldModel.fit}}\pysiglinewithargsret{\sphinxbfcode{fit}}{\emph{abstracts}, \emph{Y\_purpose}, \emph{Y\_field}}{}
Trains the model.

Input arguments must all be the same length.
\begin{description}
\item[{Args:}] \leavevmode
abstracts (list): A list of documents, each document is represented as a list of words.
Y\_purpose (list): A list of labels of the `purpose' variety.
Y\_field (list): A list of labels of the `field' variety.

\end{description}

\end{fulllineitems}

\index{get\_config() (KUHERD.Models.PurposeFieldModel method)}

\begin{fulllineitems}
\phantomsection\label{\detokenize{KUHERD:KUHERD.Models.PurposeFieldModel.get_config}}\pysiglinewithargsret{\sphinxbfcode{get\_config}}{}{}
Returns the configuration used to build this model.
\begin{description}
\item[{Returns:}] \leavevmode
dict: dictionary containing the following keys, `purpose\_vectorizer', `field\_vectorizer', `purpose\_model', `field\_model'. Each entry is the configuration required to build the model.

\end{description}

\end{fulllineitems}

\index{predict() (KUHERD.Models.PurposeFieldModel method)}

\begin{fulllineitems}
\phantomsection\label{\detokenize{KUHERD:KUHERD.Models.PurposeFieldModel.predict}}\pysiglinewithargsret{\sphinxbfcode{predict}}{\emph{abstracts}}{}
Make predictions on the input data.

The list of documents input is vectorized and input to the prediction model, which generates label predictions.
This process is done separately for generating both purpose and field label predictions.
\begin{description}
\item[{Args:}] \leavevmode
abstracts (list): A list of documents, each document is represented as a list of words.

\item[{Returns:}] \leavevmode
dictionary: dictionary containing two lists of predictions, dictionary keys are `purpose' and `field'.

\end{description}

\end{fulllineitems}


\end{fulllineitems}



\section{KUHERD.MultiFeatureSelector module}
\label{\detokenize{KUHERD:module-KUHERD.MultiFeatureSelector}}\label{\detokenize{KUHERD:kuherd-multifeatureselector-module}}\index{KUHERD.MultiFeatureSelector (module)}\index{MultiFeatureSelector (class in KUHERD.MultiFeatureSelector)}

\begin{fulllineitems}
\phantomsection\label{\detokenize{KUHERD:KUHERD.MultiFeatureSelector.MultiFeatureSelector}}\pysiglinewithargsret{\sphinxstrong{class }\sphinxcode{KUHERD.MultiFeatureSelector.}\sphinxbfcode{MultiFeatureSelector}}{\emph{scoring\_function}, \emph{kbest}, \emph{multi\_integrator}}{}
Bases: \sphinxcode{object}
\index{fit() (KUHERD.MultiFeatureSelector.MultiFeatureSelector method)}

\begin{fulllineitems}
\phantomsection\label{\detokenize{KUHERD:KUHERD.MultiFeatureSelector.MultiFeatureSelector.fit}}\pysiglinewithargsret{\sphinxbfcode{fit}}{\emph{X}, \emph{Y}, \emph{label\_set}}{}
Trains the feature selection process
\begin{description}
\item[{Args:}] \leavevmode
X (numpy matrix): Training samples.
Y (numpy matrix): Training Labels.
label\_set (str): Denotes if label set is of the `purpose' or `field' type.

\end{description}

\end{fulllineitems}

\index{transform() (KUHERD.MultiFeatureSelector.MultiFeatureSelector method)}

\begin{fulllineitems}
\phantomsection\label{\detokenize{KUHERD:KUHERD.MultiFeatureSelector.MultiFeatureSelector.transform}}\pysiglinewithargsret{\sphinxbfcode{transform}}{\emph{X}}{}
Tranforms the data by selecting the features learned in the training or \sphinxquotedblleft{}fit\sphinxquotedblright{} process.
\begin{description}
\item[{Args:}] \leavevmode
X (numpy matrix): Data samples to run feature selection on.

\item[{Return:}] \leavevmode
(numpy matrix): Data with only the selected features.

\end{description}

\end{fulllineitems}


\end{fulllineitems}



\section{Module contents}
\label{\detokenize{KUHERD:module-KUHERD}}\label{\detokenize{KUHERD:module-contents}}\index{KUHERD (module)}

\chapter{Indices and tables}
\label{\detokenize{index:indices-and-tables}}\begin{itemize}
\item {} 
\DUrole{xref,std,std-ref}{genindex}

\item {} 
\DUrole{xref,std,std-ref}{modindex}

\item {} 
\DUrole{xref,std,std-ref}{search}

\end{itemize}


\renewcommand{\indexname}{Python Module Index}
\begin{sphinxtheindex}
\def\bigletter#1{{\Large\sffamily#1}\nopagebreak\vspace{1mm}}
\bigletter{k}
\item {\sphinxstyleindexentry{KUHERD}}\sphinxstyleindexpageref{KUHERD:\detokenize{module-KUHERD}}
\item {\sphinxstyleindexentry{KUHERD.FeatureSelector}}\sphinxstyleindexpageref{KUHERD:\detokenize{module-KUHERD.FeatureSelector}}
\item {\sphinxstyleindexentry{KUHERD.HerdVectorizer}}\sphinxstyleindexpageref{KUHERD:\detokenize{module-KUHERD.HerdVectorizer}}
\item {\sphinxstyleindexentry{KUHERD.LabelTransformer}}\sphinxstyleindexpageref{KUHERD:\detokenize{module-KUHERD.LabelTransformer}}
\item {\sphinxstyleindexentry{KUHERD.Models}}\sphinxstyleindexpageref{KUHERD:\detokenize{module-KUHERD.Models}}
\item {\sphinxstyleindexentry{KUHERD.MultiFeatureSelector}}\sphinxstyleindexpageref{KUHERD:\detokenize{module-KUHERD.MultiFeatureSelector}}
\end{sphinxtheindex}

\renewcommand{\indexname}{Index}
\printindex
\end{document}